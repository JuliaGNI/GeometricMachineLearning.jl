\documentclass[tikz]{standalone}

\usetikzlibrary{fit}

\definecolor{morange}{RGB}{255,127,14}
\definecolor{mblue}{RGB}{31,119,180}
\definecolor{mred}{RGB}{214,39,40}
\definecolor{mpurple}{RGB}{148,103,189}
\definecolor{mgreen}{RGB}{44,160,44}

\usepackage{./neuralnetwork}

\newcommand{\symplecticlayer}[1][] { \layer[bias=true,nodeclass={symplectic neuron},#1] }
\newcommand{\autoencoderlayer}[1][] { \layer[bias=true,nodeclass={autoencoder neuron},#1] }

\newcommand{\xin}[2]{$x_#2$}
\newcommand{\xout}[2]{$\hat x_#2$}

\begin{document}

\begin{neuralnetwork}[height=8, layertitleheight=30, toprow=true]

  \tikzstyle{input neuron}=[neuron, fill=morange];
  \tikzstyle{hidden neuron}=[neuron, fill=mred];
  \tikzstyle{output neuron}=[neuron, fill=mgreen];
  \tikzstyle{symplectic neuron}=[neuron, fill=mblue];
  \tikzstyle{autoencoder neuron}=[neuron, fill=mpurple];

  \inputlayer[count=4, bias=false, text=\xin]

  \autoencoderlayer[count=6, bias=false]
  \linklayers[title=$\Psi^\mathrm{up}$]

  \hiddenlayer[count=6, bias=false]
  \linklayers%[title=$\psi^2_\mathrm{symp}$]
  \maketransformerblack

  \outputlayer[count=4, text=\xout]%, title=Output Layer
  \linklayers[title=$\Psi^\mathrm{down}$]
  % \node[fit=\thislayer,draw, ultra thick, rounded corners, label=left:Transformer] (transformer) {};
\end{neuralnetwork}
\end{document}
